\setcounter{page}{2}

\cleardoublepage

\section*{Abstract}

% The function of the abstract is to summarize, in one or two paragraphs, the major aspects of the entire bachelor or master thesis. It is usually written after writing most of the chapters.

% It should include the following:
% \begin{itemize}
% 	\item Definition of the problem (the question(s) that you want to answer) and its purpose (Introduction).
% 	\item Methods used and experiments designed to solve it. Try to describe it basically, without covering too many details.
% 	\item Quantitative results or conclusions. Talk about the final results in a general way and how they can solve the problem (how they answer the question(s)). 
% \end{itemize}

% Even if the Title can be a reference of the work's meaning, the Abstract should help the reader to understand in a quick view, the full meaning of the work. 
% The abstract length should be around 300 words.

% Abstracts are protected under copyright law just as any other form of written speech is protected. However, publishers of scientific articles invariably make abstracts publicly available, even when the article itself is protected by a toll barrier. For example, articles in the biomedical literature are available publicly from MEDLINE which is accessible through PubMed. It is a common misconception that the abstracts in MEDLINE provide sufficient information for medical practitioners, students, scholars and patients[citation needed]. The abstract can convey the main results and conclusions of a scientific article but the full text article must be consulted for details of the methodology, the full experimental results, and a critical discussion of the interpretations and conclusions. Consulting the abstract alone is inadequate for scholarship and may lead to inappropriate medical decisions[2].

% An abstract\cite{Ikeda1997, levensthein65:_binar, Middleton2002, salton89} allows one to sift through copious amounts of papers for ones in which the researcher can have more confidence that they will be relevant to his research. Once papers are chosen based on the abstract, they must be read carefully to be evaluated for relevance. It is commonly surmised that one must not base reference citations on the abstract alone, but the entire merits of a paper.

%Definition of the problem (the question(s) that you want to answer) and its purpose (Introduction).
Since the advancement of robotics in the modern society with robots' reliability and high work quality, robots are being used together as a multi-robot system, providing advantage compared to a single robot \cite{Eijyne2020DevelopmentOA}. 
Given an office building where people have different working schedules, if at least one person is in the office, the office is considered occupied.
%Therefore, when multiple robot working in this environment, an efficient task scheduling algorithm is required, in order to 
In this case, room occupation becomes an important evaluation parameter for task scheduling, which describes the possibility that the robot can enter the office.
Although there are increasing amounts of research have been conducted in the area of task scheduling for multi-robot system \cite{Shah7}, unfortunately, current research work mainly on multi-robot cooperation dynamic environments and rarely address on adoption of Internet of Things technologies. 
Therefore, I have considered adding more fixed sensors to in the whole office environment \cite{Coltin10}. 
In addition, I have developed a multi-robot system for this office environment, which is able to schedule the robots to gather room occupation information from fixed sensors, in order to keep room occupation information up to date.
Moreover, the system can acquire the room occupation information, and use it as one of evaluation parameters to schedule tasks, together with common evaluation parameters such as priority, battery consumption etc. 
Besides, the system utilizes an efficient communication protocol to share the information among system components.
I also pay particular attention to error handling to robust against robot failures. This system can minimize the total completion time of tasks while operating for a long period \cite{Chun12}. Simulation results demonstrate the efficiency and robustness of the task scheduling algorithm.

\cleardoublepage
