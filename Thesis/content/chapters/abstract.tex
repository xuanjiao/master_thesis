\setcounter{page}{2}

\cleardoublepage

\section*{Abstract}

% The function of the abstract is to summarize, in one or two paragraphs, the major aspects of the entire bachelor or master thesis. It is usually written after writing most of the chapters.

% It should include the following:
% \begin{itemize}
% 	\item Definition of the problem (the question(s) that you want to answer) and its purpose (Introduction).
% 	\item Methods used and experiments designed to solve it. Try to describe it basically, without covering too many details.
% 	\item Quantitative results or conclusions. Talk about the final results in a general way and how they can solve the problem (how they answer the question(s)). 
% \end{itemize}

% Even if the Title can be a reference of the work's meaning, the Abstract should help the reader to understand in a quick view, the full meaning of the work. 
% The abstract length should be around 300 words.

% Abstracts are protected under copyright law just as any other form of written speech is protected. However, publishers of scientific articles invariably make abstracts publicly available, even when the article itself is protected by a toll barrier. For example, articles in the biomedical literature are available publicly from MEDLINE which is accessible through PubMed. It is a common misconception that the abstracts in MEDLINE provide sufficient information for medical practitioners, students, scholars and patients[citation needed]. The abstract can convey the main results and conclusions of a scientific article but the full text article must be consulted for details of the methodology, the full experimental results, and a critical discussion of the interpretations and conclusions. Consulting the abstract alone is inadequate for scholarship and may lead to inappropriate medical decisions[2].

% An abstract\cite{Ikeda1997, levensthein65:_binar, Middleton2002, salton89} allows one to sift through copious amounts of papers for ones in which the researcher can have more confidence that they will be relevant to his research. Once papers are chosen based on the abstract, they must be read carefully to be evaluated for relevance. It is commonly surmised that one must not base reference citations on the abstract alone, but the entire merits of a paper.

%Definition of the problem (the question(s) that you want to answer) and its purpose (Introduction).

Since the advancement of robotics in modern society, multi-robot systems are used in office environments to help people accomplish their goals. In a typical office environment, if at least one person is in the office, the office is considered occupied. This thesis proposed a multi-robot system for an office environment. In this system, room occupation becomes an important evaluation parameter for task scheduling. However, compared with an office environment, the robot's perspective is limited, and the field within the sensing range is relatively narrow.

The knowledge of room occupation can be established by the Internet of Things (IoT) technologies. I have considered adding fixed sensors to each room in an office environment so that the robots can gather room occupation information while performing tasks. As long as the robot enters the sensor range, they conduct a rapid exchange of room occupation information. The robot then sends acquired room occupation data to the centralized pool. The centralized pool is the global controller that receives information about the robots and the environment and make decisions base on the information. 
Moreover, the centralized pool can acquire room occupation information and use it as one evaluation parameter to schedule tasks, together with common evaluation parameters such as priority, battery consumption, etc. Besides, when a robot cannot perform its current task, it hands over tasks to other robots. 

Simulation results demonstrate the efficiency and robustness of the multi-robot system. In the simulated office environment, the multi-system operated stably for days and autonomously performed thousands of navigation tasks. 


\cleardoublepage
