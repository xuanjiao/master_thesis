\setcounter{page}{2}

\cleardoublepage

\section*{Abstract}

% The function of the abstract is to summarize, in one or two paragraphs, the major aspects of the entire bachelor or master thesis. It is usually written after writing most of the chapters.

% It should include the following:
% \begin{itemize}
% 	\item Definition of the problem (the question(s) that you want to answer) and its purpose (Introduction).
% 	\item Methods used and experiments designed to solve it. Try to describe it basically, without covering too many details.
% 	\item Quantitative results or conclusions. Talk about the final results in a general way and how they can solve the problem (how they answer the question(s)). 
% \end{itemize}

% Even if the Title can be a reference of the work's meaning, the Abstract should help the reader to understand in a quick view, the full meaning of the work. 
% The abstract length should be around 300 words.

% Abstracts are protected under copyright law just as any other form of written speech is protected. However, publishers of scientific articles invariably make abstracts publicly available, even when the article itself is protected by a toll barrier. For example, articles in the biomedical literature are available publicly from MEDLINE which is accessible through PubMed. It is a common misconception that the abstracts in MEDLINE provide sufficient information for medical practitioners, students, scholars and patients[citation needed]. The abstract can convey the main results and conclusions of a scientific article but the full text article must be consulted for details of the methodology, the full experimental results, and a critical discussion of the interpretations and conclusions. Consulting the abstract alone is inadequate for scholarship and may lead to inappropriate medical decisions[2].

% An abstract\cite{Ikeda1997, levensthein65:_binar, Middleton2002, salton89} allows one to sift through copious amounts of papers for ones in which the researcher can have more confidence that they will be relevant to his research. Once papers are chosen based on the abstract, they must be read carefully to be evaluated for relevance. It is commonly surmised that one must not base reference citations on the abstract alone, but the entire merits of a paper.

%Definition of the problem (the question(s) that you want to answer) and its purpose (Introduction).

Since the advancement of robotics in modern society, multi-robot systems are used in office environments to help people accomplish their goals. In addition to interacting with people, robots also need to learn from the surrounding environment to schedule their activities effectively. 
In this research, the knowledge of room occupancy is applied to task scheduling by the robot system. Room occupation means the probability of someone in the office.  In other words, in a typical office environment,  if at least one person is in the office, the office is considered occupied. However, the perspective of individual robots cannot cover the entire office environment. But the knowledge of room occupation can be established by the Internet of Things (IoT) technologies.

There are several requirements for obtaining room occupation for the entire office environment. First, it needs to operate for a long time. Second, it needs to keep the information as current as possible. Based on these requirements, I designed a method to obtain room occupation information. I considered installing BLE sensors on all doors. The robot needs to exchange information with sensors while performing navigation tasks and find sensors when there is no navigation task. To achieve this goal, I have designed a component called the central pool. The centralized pool is the global controller in the multi-robot system. It can receive room occupation information from the robots. 
The centralized pool uses this information in two ways. It schedules the robot's navigation tasks according to the information and can also designate the robot to explore the room lacking the information.

Simulation results demonstrate the efficiency and robustness of the multi-robot system. In the simulated working environment, the multi-system operated stably for three days, and autonomously performed more than 5000 navigation tasks. This system can handle some unexpected problems. For example, the robot drops tasks when it is not able to perform them. Other robots reuse those dropped tasks.


\cleardoublepage
