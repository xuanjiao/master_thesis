

% [You should answer the question: What is the problem?]

% This paragraph should establish the context of the reported work. To do that, authors discuss over related literature (with citations\todo{How to make citations}\footnote{To cite a work in latex  }) and summarize the knowledge of the author in the investigated problem.

% An introduction should answer (most of) the following questions:
% \begin{itemize}
% 	\item What is the problem that I want to solve?
% 	\item Why is it a relevant question?
% 	\item What is known before the study?
% 	\item How can the study improve the current solutions?
% \end{itemize}

% To write it, use if possible active voice:
% \begin{itemize}
% 	\item I are going to watch a film tonight (Active voice).
% 	\item A film is going to be watched by us tonight (Passive voice).
% \end{itemize}
% The use of the first person is accepted.

\chapter{Introduction}
Since the development of robot technology in modern society, the robot's reliability and efficiency increase rapidly. Therefore, robots have been used together as a multi-robot system, which has advantages over single robots \cite{Eijyne2020DevelopmentOA}. Multi-robot systems are widely used in office environments to help people accomplish their goals. In addition to interacting with people, robots also need to learn from the surrounding environment to schedule their activities effectively. 

In this research, the knowledge of room occupancy is applied to task scheduling by the robot system.  Room occupation means the probability of someone in the office. In other words, in a typical office environment, people have different working schedules. If at least one person is in the office, the office is considered occupied. However, the perspective of individual robots is limited\cite{PYO2015148}. To be more precise, robots cannot provide heterogeneous information relevant to infer room occupation over long periods.

\section{Motivation}
This project's primary motivation is to improve the scheduling of tasks. 
Task scheduling determines when, where, and what order tasks need to be done by multiple robots. Task scheduling is an actual problem in many real-life applications, e.g., healthcare applications to assist elderly residents \cite{retire2017} inspection applications for industrial plants \cite{Chun12}. A report about the delivery robot in the hospital shows that a multi-robot system saves 2.8 full-time equivalent employees, as the robot works two shifts and seven days per week \cite{Jeon17}.
 
It then becomes essential to improve task scheduling in office environments. Task scheduling can be improved in several aspects. The first aspect is to reduce the time required to complete a set of tasks. The second aspect is to minimize the energy consumption of the robots. The knowledge of room occupation can improve task scheduling in office environments. With knowledge of room occupation, robots can find people in the environment and avoid wasting time and energy in empty rooms \cite{retire2017}. However, it is almost impossible to sense all of the necessary information using only individual robots and sensors attached to it due to their limited perspective. Therefore, it is necessary to adopt the Internet of Things (IoT) devices in office environments \cite{PYO2015148}. 




\section{Problem definition}

Our problem concerns robots that must perform various tasks in office environments. The office environment should consist of multiple rooms. In this project, I assume no robot limitation in these rooms. As long as the door of the room is opened, any robot can enter it.  The people can have their working schedules, which cause the room occupied and unoccupied regularly. I assume each room has one or multiple doors, and each door is attached to a sensor.  I consider charging stations for robots because the robot energy level drops as it moves and rotates.  

If a set of different navigation tasks randomly distributed in different rooms. The system should be able to schedule these tasks. The navigation task requires one robot to traverse a path in the office environment. Particularly, a navigation task can have one or multiple target positions. In other words, the robot can navigate to a destination or pass through several positions and finally reach the destination. Besides performing navigation tasks, robots should also gather room occupation information from fixed sensors. The system should be able to share the information among sensors, robots, and the central pool. The centralized pool is the global controller that receives information about the robots and the environment and make decisions base on the information.  Also, the robots with low-energy should go to a nearby charging station.

\section{Main Contributtions}

I have designed a method for robots to gather room occupation information while performing tasks.  As long as the robot enters the sensor range, they conduct a rapid exchange of room occupation information. The robot then sends acquired room occupation data to the centralized pool. The centralized pool is the global controller that receives information about the robots and the environment and make decisions base on the information. The centralized pool uses this information in two ways. It assigns robots the navigation tasks according to the information and can assign robots to explore the room lacking the information. Once the robot finished current tasks, it notifies the central pool that the task is succeeded and requests a task from the centralized pool.

I also paid attention to increasing the number of tasks complete.  When a robot cannot perform its current task, it hands over tasks to other robots. To be more precise, sometimes, the robot can not perform the current task within a fixed deadline since the target position is temporarily unreachable (e.g., blocked by a closed-door). In this case, the blocked robot sends failure detail to the centralized pool and request another task. The failed task will then be reused for task scheduling. 


In addition to the architecture I proposed herein, our system uses the Robot Operating System (ROS) platform to archive core autonomous robot function (navigation, localization, and mapping). Thanks to the ROS, I was able to develop a flexible and efficient system. Adding or removing modules, including robots, sensors, or charging stations, is simple and straightforward.

\section{Thesis Structure}

The next chapter briefly introduces background information. The essential concept about the Robot Operating System (ROS) and ROS Tools are introduced, along with a discussion of previous task scheduling methods.

Chapter 3  formally presents the architecture of the multi-robot system and gives information about tasks. The approaches to schedule tasks are also introduced.

Chapter 4 focuses on the implementation of each component in the multi-robot system. It presents the way robots gather room occupation information while performing tasks. 

Chapter 5 introduces the demonstrates the simulation results acquired using the approach in Chapter 3. Besides, a quantitative experiment analysis was performed.

Finally, chapter 6  discussed the results of the performed work. The limitations of the simulation are presented along with future work.



