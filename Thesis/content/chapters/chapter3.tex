\chapter{Approach}
% Describe the performed solution with all possible details. Define necessary parameters, inputs, outputs and context of use, possible problems and when they can be applied. 

% Remember to define necessary concepts before using them, building the text from easiest definitions (not depending on previous definitions) to complex definitions (depending on previous definitions).

% E.g: 
% \begin{itemize}
%	\item Lost Communication: a lost communication occurs when the conditions of the environment are not sufficient or the distance between sender and receiver is to hight to transmit information.
%	\item Wait until rescue: when the robot loses its communication, the pre-designed state machine will stop the motors to keep the actual position. Energy safe mode will be enabled, at the same time that a channel transceiver daemon will send SOS messages every T and wait for reply during T sec. 
%\end{itemize}
In our system there are multiple robots that must handle various tasks. For example, visiting given rooms. To tackle this problem, a communication efficient task scheduling system is designed. 
This system allocate task according to system resources, including environment factors, robot status and task specifications. Once this information is attained, the task scheduling system assign robot a set of task.

\begin{itemize}
	\item \textsl{Robot.} Each robot is responsible for moving in 2-dimensional physical space as well as colleting measurement result from sensors. It has a rechargeable battery, and its level drops as robot moves and rotates.
	\item \textsl{Tasks.} Each task requires one or more robots to traverse a path in the workspace and carry out certain actions\cite{Ivan2017}.
	\item \textsl{Environment.} In this project, all robots are considered moving in an office environment that contains a corridor along the central x-axis and 16 rooms located around the corridor. The environment model is shown in Figure \ref{fig:gazebo_model}. The environment factors, such as room locations and occupancy possibilities help task allocation.
\end{itemize}

\begin{figure}[htbp]
	\centering
	\includegraphics[width = 0.9\textwidth]{content/images/ch3/gazebo_model.png}
	\caption{Gazebo Model}
	\label{fig:gazebo_model}
\end{figure}

\section{Architecture Design}

The architecture of the system consist of several parts: centralized pool, robot controller, navigation stack, charging station and system environment(Figure \ref{fig:system_architecture}). 
\begin{itemize}
	\item \textsl{Centralized Pool.} A centralized pool consist of several modules: multi-robot task allocation module, map information, database, execution and monitoring. The database contains most of the environment information(Figure \ref{fig:database_er}). The multi-robot task allocation module allocate tasks to robots once requested.
	\item \textsl{Robot Controller.} A robot controller contains several modules: local task queue, execution and robot action. The execution module receives commands from centralized pool and decides when and which task the robot should perform.
	\item \textsl{Navigation stack.} The move\_base node provides a ROS interface for configuring, running, and interacting with the navigation stack on a robot. It makes robot move to desired positions using the navigation stack. Its advantages include optionally performing recovery behaviors when the robot perceives itself as stuck\cite{move_base_node}. 
\end{itemize} 

\begin{figure}[htbp]
	\centering
	\includegraphics[width = 0.9\textwidth]{content/images/ch3/architecture.drawio.png}
	\caption{System Architecture}
	\label{fig:system_architecture}
\end{figure}

\begin{figure}[htbp]
	\centering
	\includegraphics[width = 0.9\textwidth]{content/images/ch3/occupancy_grid.png}
	\caption{Environment Occupancy Grid}
	\label{fig:occupancy_grid}
\end{figure}

\section{System Environment}
In this project, it is assumed that there are no dynamic obstacles. As is shown in Figure \ref{fig:occupancy_grid}, the black lines are occupied area, which is the wall in 3D-Model(Figure \ref{fig:gazebo_model}). The gray area is unknown area. 
The white area is unoccupied. In unoccupied area there are following important areas and coordinates:
\begin{itemize}
	\item \textsl{Rooms.} The rectangle areas (Figure \ref{fig:room_division}) are used to represent Rooms. Each rectangle has its upper and lower limit in x and y coordinates, in order to identify which rooms the robots or targets belong to. 
	\item \textsl{Doors.} The positions of doors (Figure \ref{fig:positions_door_station}) are stored in database. There are used by a ROS door simulator node, which broadcasts positions and door status periodically. The broadcast messages are received and filtered by robots.
	\item \textsl{Charging Stations.} The positions of charging stations are used by ROS charging station nodes. For details please refer to Section \ref{sec:charging_station}.
\end{itemize}

\begin{figure}[htbp]
	\centering
	\includegraphics[width = 0.9\textwidth]{content/images/ch3/room_division.png}
	\caption{Room division}
	\label{fig:room_division}
\end{figure}

\begin{figure}[htbp]
	\centering
	\includegraphics[width = 0.9\textwidth]{content/images/ch3/positions_door_station.png}
	\caption{Doors and Charging Stations}
	\label{fig:positions_door_station}
\end{figure}

\section{Task Allocation}

\subsection{Task Specifications}
\label{sec:task_specifications}
In order to improve overall execution efficiency, one single robot can carry either small task, which make robot move to one position, or large task, which consist of multiple small task.
On one hand, recharging is necessary for robots to work long hours. On the other hand, a robot should gather environment information as much as possible, which centralized pool would learn from and make better decision. 
Therefore, three types of task are defined. The task specifications are various for different task types (Table \ref{tab:task_specifications_difference} and Table \ref{tab:task_specifications_common}).

\begin{table}[htb]
\centering
\resizebox{\textwidth}{!}{
\begin{tabular}{|c|c|c|c|c|c|c|c|c|c|} 
\hline
\diagbox{Task type}{Task Specifications} & Target & Dependency & Priority & Generator \\
\hline
GatherEnvironmentInfo Task & Door &  No & 1 & Centralized Pool\\
\hline
Execute task & Any point & Dependent on Execute-task & 2,3,4 & User\\
\hline
Charging Task & Charging station & No & 5 & Centralized Pool\\ [1ex] 
\hline
\end{tabular}}
\caption{Task Specifications Part 1}
\label{tab:task_specifications_difference}
\end{table}

\begin{table}[htb]
\centering
\resizebox{\textwidth}{!}{
\begin{tabular}{|c|c|c|} 
\hline
Task Specifications	& Start Time &	Finish Time \\
\hline
Explanation	& The time when robot start moving &	The time when robot finished interacting with the target \\ [1ex] 
\hline
\end{tabular}}
\caption{Task Specifications Part 2}
\label{tab:task_specifications_common}
\end{table}


\begin{table}[htb]
\centering
\resizebox{\textwidth}{!}{
\begin{tabular}{|c|c|c|c|c|c|c|c|c|c|} 
\hline
Task Id & Task Type & Start Time & Target Id & Robot Id & Priority & Status & dependency & Finish Time  & Description \\
\hline
1 & 2 & 10:00:00 & 10:59:59 & 0.80 & 2 & RanToCompletion & 0 & 2020-06-01 9:00:00 & Succeeded\\ [1ex] 
\hline
\end{tabular}}
\caption{Task Table in Database}
\label{tab:task_table}
\end{table}

\begin{itemize}
	\item \textsl{Task Size.} The task in database (Table \ref{tab:task_table}) is referred to small task. The example of small task includes "charging task" and "gather environment information task". 
	Those small tasks form a dependency chain, also referred to a large task. Execute task can be a large task which ask a robot to move continuously to several positions.
	\item \textsl{Target.} Targets include doors, points and charging stations. When a robot run a "gather environment information task", it moves to the front of a door and interact with a sensor in the door position without entering the door. 
	When robot run an "execute task", the robot moves to a given point ether in corridor or in the room.
	When robot run a "charging task", the robot moves to a charging station and interact with this charging station.
\end{itemize}

\subsection{Execute Task Allocation}
\label{sec:task_allocation}
With robot status such as positions and available battery provided by robot, the multi-robot task allocation module in the architecture should perform multi-robot task allocation. To select an "execute task", the following decision variables are considered.

\paragraph*{Decision variables}

\begin{table}[htb]
\centering
\begin{tabular}{|c| c| c|} 
\hline
Door Id & Door Status & Date Time \\
\hline
1& 1 & 2020-06-01 09:00:01 \\ [1ex] 
\hline
\end{tabular}
\caption{Measurement Result}
\label{tab:measurement_result}
\end{table}

\begin{table}[htb]
\centering
\resizebox{\textwidth}{!}{
\begin{tabular}{|c|c| c| c| c| c|} 
\hline
Door id & Day Of Week & Start Time & End Time & Init Open Possibility &  Open Possibility Statistic \\
\hline
1 & 2 & 10:00:00 & 10:59:59 & 0.80 & 0.80 \\ [1ex] 
\hline
\end{tabular}}
\caption{Door Open Possibility}
\label{tab:open_possibilities}
\end{table}

\begin{itemize}
\item \textsl{Task Priority.} Task Priority. Task priority is an important factor that describes task emergency level. The "charging task" has the highest priority of 5. The "gather environment information" task has the lowest priority of 1. 
	The "execute task" is determined by users but must be in the range of [2,4]. 
\item \textsl{Product of Door Open Possibility.} Because of the limitation of simulation, the door open possibilities are used to represent room occupancies. A door open possibility is based on the statistic of door measurement results in a specific time period of a working day. 
	The doors that the robot may pass through can be obtained from the map information module.
	An example of "measurement result" table is shown in Table \ref{tab:measurement_result}, an example of "open possibility" table is shown in Table \ref{tab:open_possibilities}. 
\item \textsl{Waiting Time. } The waiting time is the difference between the current simulation time and start time of the first task to be executed. $T_{waiting} = T_{first\_task} - T_{now}$
\item \textsl{Battery Consumption.} The Battery Consumption is related to robot trajectory. For a Large "execute task" that contains n small task, Equation \ref{eq:battery_consumption} can be used to calculate battery consumption. The centralized pool will send the task with the lowest cost to this robot.
\end{itemize}
\begin{equation}
\begin{aligned}
\label{eq:battery_consumption}
&B: Battery\_Consumption W: Weight\\
B_{large\_task} & = \sum_{task_0}^{task_n} B_{trajectory} \\
& = \sum_{task_0}^{task_n} \sum_{waypoint_0}^{waypont_M} [W_{position} \times position\_variation+W_{angle}  \times angle\_variation]\\
& = \sum_{t = task_0}^{task_n} \sum_{p = waypoint_0}^{waypoint_M} [ W_{position} \times \sqrt{(x_p-x_{p-1} )^2+(y_p-y_{p-1} )^2} \\
&   + W_{angle} \times 2 \times \arccos(w_p)] 
\end{aligned}
\end{equation}

In conclusion, equation \ref{eq:large_execute_task_cost} can be used to calculate the cost of a large "execute task".

\begin{equation}
	\label{eq:large_execute_task_cost}
	\begin{split}
	Cost_{Large\_execute\_task} = \frac{W_{battery} \times Battery\_consum}{n} + W_{waiting} \times Waiting\_time \\
	+ W_{possibility} \times \prod\limits_{i=1}^n Door\_open\_possibility  + W_{priority} \times Priority
	\end{split}
\end{equation}


\subsection{Environment Task Allocation}
Once robot request a task, The task allocation module select only tasks with cost below the threshold. Once either no task in database or all costs above the threshold, 
the task allocation module should create a "gather environment information task", in order to collect more measurement result and further more improve the accuracy of "open possibilities" table.
To create a "gather environment information task", which door should the requesting robot visit must be considered. The following decision factors help task allocation module to select the door.

\paragraph*{Decision variables}
\begin{itemize}
	\item \textsl{Door Last Update Time.} The latest timestamp when the door is measured.
	\item \textsl{Battery Consumption.} Similar to "execute task" allocation, the battery consumption is related to the trajectory from robot to the door. Equation \ref{eq:battery_consumption} can be used to calculate battery consumption.
	\item \textsl{Whether door is used.} If another robot is going to this door, the value is 0, otherwise the value is 1.
\end{itemize}

\begin{equation}
	\label{eq:door_cost}
	\begin{split}
	Cost_{door} = \frac{W_{battery} \times Battery\_consum}{n} + W_{time} \times (T_{last\_update} - T_{now}) \\
	+ W_{possibility} \times \prod\limits_{i=1}^n Door\_open\_possibility + W_{is\_used} \times Is\_used  
	\end{split}
\end{equation}

\subsection{Charging Task Allocation}
Once a robot sends task request to the centralized pool, the centralized pool figures out whether this robot need charging, if yes it should create a "charging task" for requsting robot (Figure \ref{fig:centralized_task_allocation}). Since there are multiple charging station in the system environment (Figure \ref{fig:positions_door_station}), the centralized pool selects a charging station for this robot using the following decision variables.

\paragraph*{Decision variables}

\begin{itemize}
	\item \textsl{Remain Time.} It describes how long will a charging station be free. 
	\item \textsl{Battery Consumption.} Similar to "execute task" allocation, the battery consumption is related to the trajectory from robot to the charging staion. Equation \ref{eq:battery_consumption} can be used to calculate battery consumption.
\end{itemize}
In conclusion, equation \ref{eq:charging_station_cost} can be used to calculate the cost of a charging station. The centralized pool will generate a "charging task" for requesting robot to charging station with the lowest cost.

\begin{equation}
	\label{eq:charging_station_cost}
	\begin{split}
	Cost_{charging\_station} = \frac{W_{battery} \times Battery\_consum}{n} + W_{time} \times T_{remain}
	\end{split}
\end{equation}

 % + wt_wait \times waiting time + \prod\limits_{i=1}^n open possibility+wt_pri × priority
% wt_btr × battery consumption ÷ N + wt_wait × waiting time+ 


