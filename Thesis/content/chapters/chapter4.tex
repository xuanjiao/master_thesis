\chapter{Implementation}

%Explain what you did to implement your solution, problems that occurred and how you fixed them. 
%If they are interesting, include some relevant parts of the implementation (most relevant pieces of code and so on). 

\section{Communication Protocols}


Centralized pool and robots need to share task information with each other. There are some basic requirements for communication: firstly,
robot should initiate the communication once it has finished all task in task queue and get free. This is solved by assigning robot controller as ROS service client and centralized pool as ROS service server.
This method saves unnecessary communication cost by avoiding keep tracking the current position, availability and states of all robots.
Secondly, robot should forward sensor data to centralized pool while processing a task. This is solved by assigning robot controller as ROS action server and centralized pool as ROS action client.
As is shown in \ref{fig:comminication}, an efficient communication protocols is designed. 

\subsection{Message Format}
Four types of message are defined: 
(1)Task request message(Table\ref{tab:request_message}); (2) Task goal messages(Table \ref{tab:goal_message}); (3) Task feedback message (Table \ref{tab:feedback_message}); (4) Task result message (Table \ref{tab:result_message}). 

\begin{figure}[htbp]
    \centering
    \includegraphics[width = 0.7\textwidth]{content/images/ch4/robot_pool_comminication.drawio.png}
    \caption{Communication between Robot and Centralized Pool}
    \label{fig:comminication}
\end{figure}

\begin{table}[htb]
\centering
\begin{tabular}{|c|c|c|} 
\hline
Battery & Pose & Robot ID\\
\hline\hline
93	&(2,4)	&1 \\ [1ex] 
\hline
\end{tabular}
\caption{Request Message Format and Example}
\label{tab:request_message}
\end{table}

\begin{table}[htb]
\centering
\resizebox{\textwidth}{!}{
\begin{tabular}{|c|c|c|c|} 
\hline
Task id[] &Task type & Target id & Goal[] \\
\hline\hline
1& Gather Environment Info & 9	& (-1.5,5.2) 2020-06-01 9:00:00 \\
\hline
[3,4]	& Execute task & 21, 22	& (-24.0,12.0), 2020-06-01 9:02:00 (-21.0,12.0) 2020-06-01 9:02:00 \\
\hline
5	& Charging	& 17	&(0.0,5.0), 2020-06-01 9:04:00 \\ [1ex] 
\hline
\end{tabular}}
\caption{Action Goal Message Format and Example}
\label{tab:goal_message}
\end{table}

\begin{table}[htb]
\centering
\begin{tabular}{|c|c|c|c|} 
\hline
Robot id & Door id & Measurement time & Measurement result \\
\hline\hline
1	& 3	& 2020-06-01 9:00:03 & Door open \\ [1ex] 
\hline
\end{tabular}
\caption{Action Feedback Message Format and Example}
\label{tab:feedback_message}
\end{table}

\begin{table}[htb]
\centering
\begin{tabular}{|c|c|c|} 
\hline
Task id	& Task type	& Result\\
\hline\hline
1 & Gather Enviroment Info & Success \\ [1ex] 
\hline
\end{tabular}
\caption{Action Result Message Format and Example}
\label{tab:result_message}
\end{table}


\begin{figure}[htbp]
    \centering
    \includegraphics[width = 0.7\textwidth]{content/images/ch4/database_er.png}
    \caption{Database Entity Relationship Diagram}
    \label{fig:database_er}
\end{figure}


\section{Database}
The centralized pool keep information it requires, to make decisions. Figure \ref{fig:database_er} shows the relationship between entities. Following are explanations of some tables.
\begin{itemize}
	\item \textsl{Table doors.} Table doors stores enviroment information. In table doors, the is\_used column will be updated when an Enviroment-task to this door is updated. The last update will be updated when the centralized pool receive a new measurement result. 
	\item \textsl{Table open\_ossibilities.} Table open\_ossibilities is based on the statistic of door measurement in a specific time period of each working day. These two table will be updated when centralized pool receive a new measurement result. 
	\item \textsl{Table exe\_rs.} Table exe\_rs stores experiment result while table exe\_weight, table door\_weight and table charging\_station\_weight stores weight values for experiment. Chapter 5 introduces the details of experiment.
	\item \textsl{Table costum\_points.} When user gives the system a task, the target point of this task will be stored in position table, and a target\_id will be generated.  This target\_id will be stored in custum\_point table. 	Additionally, with the information in room\_range table, the system will recognize which room does the point belong to and write room\_id column in costom\_points table.
\end{itemize}


\section{Procedure}
As stated in the Chapter 3, the goal of task scheduling is finishing all tasks as soon as possible while keep the cost as low as possible. 
The task assignment and execution happends at two level. \cite{Ivan2017} the task and the path planner solves a planning problem. It takes and occupancy grid, a specific robot and a set of task specifications, and generates trajectorys for each task. According to those trajectorys and task specifications, the task with the lowest cost is assigned to robot.
At the dynamic level, after each robot receive a task, it runs a navigation stack to execute this task stepwise. Each robot computes a local trajectory but takes into account dynamic obstacles.
The process of the robot task allocation system is as follows.


\begin{figure}[htbp]
    \centering
    \includegraphics[width = 0.7\textwidth]{content/images/ch4/centralized_task_select.drawio.png}
    \caption{Task allocation in Centralized Pool}
    \label{fig:centralized_task_allocation}
\end{figure}
\subsection{Centralized Pool}

Once the centralized pool receives a task request from robot, it will perform task allocation. The task allocation algorithm is discussed in Section \ref{sec:task_allocation}. The process of task allocation is shown in Figure \ref{fig:centralized_task_allocation}. 
\begin{enumerate}
	\item When the battery of robot belows 10\%, the centralized pool create a charging task to the charging staion with the lowest cost.
	\item When the battery of robot aboves 10\%, the centralized pool loads execute-tasks in database, then combine small tasks with dependencies into large tasks, finally calculates task costs and select one large task with the lowest cost. 
	\item If there are no suitable tasks, a gather-enviroment-task to the door with the lowest cost is generated. 
\end{enumerate}
The difference between task is discussed in Section \ref{sec:task_specifications}. The output of the task allocation includes: task ID, goal coodinate, timestamp and selected robot ID. The task is sent to the selected robot, and the robot performs the tasks.





\subsection{Robot}
\paragraph{Robot Structure} The structure of robot is shown in Figure \ref{fig:robot_components}. 

\begin{figure}[htbp]
	\centering
	\includegraphics[width = 0.45\textwidth]{content/images/ch4/system_component_robot.drawio.png}
	\caption{Robot Components}
	\label{fig:robot_components}
\end{figure}

\begin{itemize}
	\item \textsl{Local task queue.} Local task queue keeps a list of tasks that a robot will run sequentially. Once a task is finished, it would be removed from task queue. Once this queue become empty, the robot send task result to centralized pool.
	\item \textsl{New task client.} Once all task are finished, the new task client send request to new task server.
	\item \textsl{Run task server.} The run task server receive tasks and send task feedback and task result.
	\item \textsl{Timer.} To prevent robot to be hanged by one task forever, the timer check the robot moving state periodically.
	\item \textsl{Move base client.} The move\_base node provides a ROS interface for configuring, running, and interacting with the navigation stack on a robot. The move\_base client send a goal to move\_base node to tracking their status  
	\item \textsl{Position subscriber.} The position subscriber get robot current position from navigation stack. The robot send its current location to centralized pool to request a new task.
	\item \textsl{Sensor subscriber.} The Sensor subscriber listen to sensor data within the range.
\end{itemize}

\paragraph{Robot Task Processing} The flowchart of robot task processing is shown in Figure \ref{fig:robot_task_processing}.

\begin{figure}[htbp]
    \centering
    \includegraphics[width = 0.7\textwidth]{content/images/ch4/robot_task_flow.drawio.png}
    \caption{Robot Task Processing Flowchart}
    \label{fig:robot_task_processing}
\end{figure}






\subsection{Charging Station}
\label{sec:charging_station}