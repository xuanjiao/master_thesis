\chapter{Implementation}

%Explain what you did to implement your solution, problems that occurred and how you fixed them. 
%If they are interesting, include some relevant parts of the implementation (most relevant pieces of code and so on). 

\section{Communication Protocols}

\begin{figure}[htbp]
	\centering
	\includegraphics[width = 0.45\textwidth]{content/images/ch3/communication_protocals.drawio.png}
	\caption{Communication Protocols}
	\label{fig:communication_protocals}
\end{figure}

Since centralized pool and robots need to share task information with each other, the communication protocals are required. Four types of message are defined: (1) task request message; (2) task goal messages; (3) task feedback; (4) task result. 
The comparation of task type and examples of each type are shown in Figure \ref{fig:communication_protocals}. 

\section{Database structure}

\section{Procedure}

each robot autonomously request task from the centralized pool and centralized pool response with a set of suitable tasks. 
\subsection{Robot}


\begin{figure}[htbp]
	\centering
	\includegraphics[width = 0.45\textwidth]{content/images/ch3/system_component_robot.drawio.png}
	\caption{Robot Components}
	\label{fig:robot_components}
\end{figure}

\begin{itemize}
	\item \textsl{Robot ID.} Robot ID is a unique identification for each robot.
	\item \textsl{Battery level.} Battery level drops as the robot moves and rotates.
	\item \textsl{Task type.} Robots perform different tasks such as "Charging", " Execute Task", "Gather Enviroment Information". For details please refer to Section \ref{sec:task_types}
	\item \textsl{Local task queue.} Local task queue keeps a list of tasks that a robot will run sequentially. Once a task is finished, it would be removed from task queue. Once this queue become empty, the robot send task result to centralized pool.
	\item \textsl{New task client.} Once all task are finished, the new task client send request to new task server.
	\item \textsl{Run task server.} The run task server receive tasks and send task feedback and task result.
	\item \textsl{Timer.} To prevent robot to be hanged by one task forever, the timer check the robot moving state periodically.
	\item \textsl{Move base client.} The move\_base node provides a ROS interface for configuring, running, and interacting with the navigation stack on a robot. The move\_base client send a goal to move\_base node to tracking their status  
	\item \textsl{Position subscriber.} The position subscriber get robot current position from navigation stack. The robot send its current location to centralized pool to request a new task.
	\item \textsl{Sensor subscriber.} The Sensor subscriber listen to sensor data within the range.
\end{itemize}

\subsection{Centralized Pool}

\begin{figure}[htbp]
	\centering
	\includegraphics[width = 0.45\textwidth]{content/images/ch3/system_component_centralized_pool.drawio.png}
	\caption{Centralized Pool Components}
	\label{fig:centralized_pool_components}
\end{figure}

\begin{itemize}
	\item \textsl{Map Information.} Map information contains information such as the door list that the robot will pass through when moving to target position.
	\item \textsl{Cost calculator.} Cost calculator calculate the cost for doors, rooms and charging stations.
	\item \textsl{Task manager.} Task manager can construct, sort and allocate tasks.
\end{itemize}





\subsection{Charging Station}