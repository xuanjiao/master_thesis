\chapter{Evaluation}

In this chapter you should describe the previous (if possible) and final experiments performed on the implementation.

Every single experiment should be explained individually, providing to the reader information about the meaning of the experiment, the expected (theoretical) results, the final results, the comparison between them and others (if possible) and the conclusions. 

Each experiment should include a description, covering (when possible) the following information:
\begin{itemize}
	\item Significant physical features (obstacles present on the environment, human presence, temperature, humidity, possible noise sources, computational speed of the machine, etc.)
	\item The precise location of the experiment (latitude and longitude, room number or citation to a description of the used laboratory).
	\item Sampling design (variable(s) measured, transformation performed to the data, samples collected, replication, comparative with a Ground Truth system, collecting data protocol).
	\item Analysis design (how the data are processed, statistical procedures used, statistical level to determine significance).
\end{itemize}
The provided information should be sufficient to allow other scientists to repeat your experiment in the same conditions. Thus, the use of standard and well-known equipment could only be represented by a simple sentence, but the non-standard equipment should be described in detail, citing the source (vendor) and most important characteristics.

To write it, try to use the third person when describing the experiments and results. Avoid to use first person. Past tense should be the dominant conjugation (the work is done and was performed in the past).

Note: Graphics represent really well data, use them! (Matlab or Octave could be useful for that).
